% writeup 

\documentclass[12pt]{report}


\usepackage{amsmath}
\usepackage{graphicx}
\usepackage{caption}
\usepackage{mathtools}
\usepackage{cleveref}
\usepackage{courier}

\graphicspath{{./figs/}}

\begin{document}

\title{Complexity Project: The Oslo Model}
\author{Spencer Wilson}
\date{\today}
\maketitle

\newpage


% Size limit: Word limit for each report: 2500 words. Write no. words used at the front of each report. Use 12pt font, minimum 2.5cm left and right margins. Max 16 pages, excluding front page and ultimate acknowledgment + bibliography page.

% The reports should include:
% brief descriptions of the aims and methods a summary of the results
% a brief discussion
% a conclusion.

% Your focus will depend on how far you get through the tasks. Marks will be awarded for:
% -- correct results achieved; understanding of physics and underlying theory; sensible presentation, interpretation and explanations of the results, soundness of conclusions drawn (weight 75%)
% -- organisation, general quality & presentation of report (weight 25%).

\section*{1}

Devise and perform some simple tests (e.g. by selecting particular simple values of p) to check whether your programme is working as intended.

We plug in for p=1, p=0 

We expect half and double recurrent heights, with single recurrent 
%
\begin{align*}
P^T\textbf{x} &= \lambda{\textbf{x}} \\ 
\textbf{x}^TP^T\textbf{x} &= \textbf{x}^T\lambda{\textbf{x}} \\
\frac{\textbf{x}^TP^T\textbf{x}}{\textbf{x}^T\textbf{x}} &= \lambda 
\end{align*}
%
\begin{equation*}
P^T\textbf{x} = \textbf{x}	
\end{equation*}
%
\begin{figure}[htbp]
\begin{center}
\includegraphics[width=\textwidth]{heightsTest.png}
\caption{}
\label{heightsTest}
\end{center}
\end{figure}
%
\begin{figure}[htbp]
\begin{center}
\includegraphics[width=\textwidth]{sizetimeseries}
\caption{}
\label{sizetimeseries}
\end{center}
\end{figure}
%
\begin{figure}[htbp]
\begin{center}
\includegraphics[width=\textwidth]{sizes_logbin}
\caption{}
\label{sizes_logbin}
\end{center}
\end{figure}

%%%%%%%%%%%%%%%%%%%%%%%%%%%%%%%

\section*{2}

%%%%%%%%%%%%%%%%%%%%%%%%%

QUESTIONS

-is the function F just something*h(argument)?
	(1+1/L)h(L)
	SOLVED 

-check collapsed height plot -- this good enough? 
	stretch the data out to show the higher system sizes 
	-why do the recurrent heights go down slightly for lower system sizes? 
	CHECK 

-check collapsed gaussian -- mean is moved over?
	scale the x and y axis the same way 
	SOLVED (with binning)
		-- IS THIS OK? 

-check the correction to recurrent height mean scaling 
	argument about w1 = 1 (1/L), a1 = a0/intercept 
	different fitting methods, slightly different 
		log of both sides, etc etc 
	SOLVED -- iterate over a0 

-- how does standard deviation scale with height? 
	ASK 


%%%%%%%%%%%%%%%%%%%%%%%%%

% (a)

% - Starting from an empty system, measure the total height of the pile as a function of time t for the range of system sizes listed above. Plot the height h(t; L) vs. time t for the various system sizes in the same plot. 

% 	DONE 

% - Reflect upon the results obtained in terms of transient and recurrent configurations. 

% 	DONE 

% - How does the cross-over time $t_c(L)$ for the system to reach the recurrent configurations (steady state) scale with system size? Qualify your answer. 

% 	HOW DO WE GET THIS ACCURATELY? FIT DATA FOR T_C, WE EXPECT ~L^2 ?
% 		--fit quadratically? why quadratic? 
% 				write down tc v L for p=0 and p=1 
% 				think about the area of the triangle  

% - How does the height of the pile in the steady state scale with system size? Qualify your answer.

% 	LINEARLY? PLOT THIS 
% 	mean(heights[-100:-1 ]

% (b) 

% MEASURE (for t0 > tc):

% 	average height
% 	std of the height 
% 	height probability 

% - Guided by your answers to the two questions in TASK 2a, produce a data collapse for the processed height ~h(t; L) vs. time t for the various system sizes. 

% 	HOW? 

% - Explain carefully how you produced a data collapse and express that mathematically, introducing a scaling function F: ~h(t; L) = something*F(argument), identifying ‘something’ and the ‘argument’. 

% - How does the scaling function F(x) behave for large arguments and for small arguments and why must it be so? 

% - From this result, obtain/predict how ~h(t;L) increases with t during the transient.

\end{document}