% writeup 

\documentclass[12pt]{report}


\usepackage{amsmath}
\usepackage{graphicx}
\usepackage{caption}
\usepackage{mathtools}
\usepackage{cleveref}
\usepackage{courier}

\graphicspath{{./figs/}}

\begin{document}

\title{Complexity Project: The Oslo Model}
\author{Spencer Wilson}
\date{\today}
\maketitle

\newpage

\section*{1}

Devise and perform some simple tests (e.g. by selecting particular simple values of p) to check whether your programme is working as intended.

We plug in for p=1, p=0 

We expect half and double recurrent heights, with single recurrent 
%
\begin{align*}
P^T\textbf{x} &= \lambda{\textbf{x}} \\ 
\textbf{x}^TP^T\textbf{x} &= \textbf{x}^T\lambda{\textbf{x}} \\
\frac{\textbf{x}^TP^T\textbf{x}}{\textbf{x}^T\textbf{x}} &= \lambda 
\end{align*}
%
\begin{equation*}
P^T\textbf{x} = \textbf{x}	
\end{equation*}
%
\begin{figure}[htbp]
\begin{center}
\includegraphics[width=\textwidth]{heightsTest.png}
\caption{}
\label{heightsTest}
\end{center}
\end{figure}
%
\begin{figure}[htbp]
\begin{center}
\includegraphics[width=\textwidth]{sizetimeseries}
\caption{}
\label{sizetimeseries}
\end{center}
\end{figure}
%
\begin{figure}[htbp]
\begin{center}
\includegraphics[width=\textwidth]{sizes_logbin}
\caption{}
\label{sizes_logbin}
\end{center}
\end{figure}

%%%%%%%%%%%%%%%%%%%%%%%%%%%%%%%

\section*{2}

%%%%%%%%%%%%%%%%%%%%%%%%%

QUESTIONS

-is the function F just something*h(argument)?
-check collapsed height plot
-check collapsed gaussian
-check the correction to recurrent height mean scaling 
	argument about w1 = 1 (1/L)
	how to fit (all data or only 128-256)


%%%%%%%%%%%%%%%%%%%%%%%%%

% (a)

% - Starting from an empty system, measure the total height of the pile as a function of time t for the range of system sizes listed above. Plot the height h(t; L) vs. time t for the various system sizes in the same plot. 

% 	DONE 

% - Reflect upon the results obtained in terms of transient and recurrent configurations. 

% 	DONE 

% - How does the cross-over time $t_c(L)$ for the system to reach the recurrent configurations (steady state) scale with system size? Qualify your answer. 

% 	HOW DO WE GET THIS ACCURATELY? FIT DATA FOR T_C, WE EXPECT ~L^2 ?
% 		--fit quadratically? why quadratic? 
% 				write down tc v L for p=0 and p=1 
% 				think about the area of the triangle  

% - How does the height of the pile in the steady state scale with system size? Qualify your answer.

% 	LINEARLY? PLOT THIS 
% 	mean(heights[-100:-1 ]

% (b) 

% MEASURE (for t0 > tc):

% 	average height
% 	std of the height 
% 	height probability 

% - Guided by your answers to the two questions in TASK 2a, produce a data collapse for the processed height ~h(t; L) vs. time t for the various system sizes. 

% 	HOW? 

% - Explain carefully how you produced a data collapse and express that mathematically, introducing a scaling function F: ~h(t; L) = something*F(argument), identifying ‘something’ and the ‘argument’. 

% - How does the scaling function F(x) behave for large arguments and for small arguments and why must it be so? 

% - From this result, obtain/predict how ~h(t;L) increases with t during the transient.

\end{document}